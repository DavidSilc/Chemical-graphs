\documentclass{article}
\setlength{\parindent}{0pt}
\hyphenpenalty=1000
\tolerance=1000
\sloppy



\begin{document}

\section*{Skupina 18: Kemijski grafi}
\textit{Avtorja: David Planinšek Šilc, Lenart Žerdin \\ Datum: 20. 12. 2024 \\}


Najina tema je raziskovanje kemijskih grafov in njihovega totalnega $\sigma$-indeksa iregularnosti.

\subsection*{Opis problema}

Graf je kemijski, če so vsa njegova vozlišča stopnje največ 4. Če ima kemijski graf $a_i$ vozlišč stopnje $i$, $1 \leq i \leq 4$, potem njegovo stopenjsko zaporedje označimo kot $(1^{a_1}, 2^{a_2}, 3^{a_3}, 4^{a_4})$. \\

Definiramo totalni $\sigma$-indeks iregularnosti, v angleščini `Total $\sigma$-irregularity', $\sigma_t^{f(n)}(G)$ kot
\[
\sigma_t^{f(n)}(G) = \sum_{\{u,v\} \subseteq V(G)} \left| d_G(u) - d_G(v) \right|^{f(n)},
\]
kjer je $n = |V(G)|$ in je $f(n)$ funkcija, definirana za $n \geq 4$. \\


Očitno je, da je minimalna vrednost $\sigma_t^{f(n)}$ dosežena z regularnimi grafi, ki jih na primer ponazarjajo strukture, kot so cikel $C_n$ ali grafi z stopnjskimi zaporedji, kot so $(1^0, 2^0, 3^0, 4^n)$. 
Zato se osredotočamo na kemijske grafe, ki dosegajo maksimalno vrednost $\sigma_t^{f(n)}$.

\subsubsection*{Izrek}
Naj bo $n \geq 7$, $f(n) \leq \log_3 \left( \frac{3n^2}{3n^2 - 8} \right)$, in naj bo $(1^{a_1}, 2^{a_2}, 3^{a_3}, 4^{a_4})$ stopenjsko zaporedje kemijskega grafa $G$ z maksimalno vrednostjo $\sigma_t^{f(n)}(G)$. Potem velja:
\begin{enumerate}
    \item Če $n = 4k - 1$, potem $a_1 = a_3 = a_4 = k$ in $a_2 = k - 1$.
    \item Če $n = 4k$, potem $a_1 = a_2 = a_3 = a_4 = k$.
    \item Če $n = 4k + 1$, potem $a_1 = a_2 = a_3 = k$ in $a_4 = k + 1$.
    \item Če $n = 4k + 2$, potem velja bodisi $a_1 = a_3 = k$ in $a_2 = a_4 = k + 1$, bodisi $a_1 = a_3 = k + 1$ in $a_2 = a_4 = k$. 
\end{enumerate}

Ker je za kemijske grafe razlika med stopnjami vozlišč omejena, domnevamo naslednje: 

\begin{itemize}
    \item \textbf{Domneva 1:} Isti grafi, kot v Izreku, imajo maksimalno vrednost za $\sigma_t^{f(n)}$, če je $f(n) = \frac{1}{n}$. 
    \item \textbf{Domneva 2:} Isti grafi, kot v Izreku, imajo maksimalno vrednost za $\sigma_t^{f(n)}$, če je $f(n) = c$, kjer je $c$ konstanta v intervalu $(0, 1)$. 

\end{itemize}


\subsection*{Potek dela}


\begin{enumerate}
    \item Implementirala bova algoritmem za sistematično iskanje vseh možnih stopenjskih zaporedij za 
    kemijske grafe do števila vozlišč $n=10$, saj se čas iskanja eksponentno povečuje.
    \item Za vsako stopenjsko zaporedje bova izračunala $\sigma_t^{f(n)}(G)$ z ustrezno funkcijo $f(n)= \frac{1}{n}$
     ali $f(n)= c$, $c \in (0,1)$, in jih uredila po velikosti. Osredotočila se bova na $c$ -je, ki so blizu $0$ oziroma $1$.
    \item Za obe funkciji in za različne parametre $c$ bova preverila ujemanje s 
    posledicama izreka. 
    \item Implementirala bova algoritmem za stohastično iskanje stopnejskih zaporedij za kemijske grafe za $n>10$.
    Najina hipoteza je, da morajo biti za večji totalni $\sigma$-indeks števila stopenjskega zaporedja blizu skupaj. 
    Algoritem bo začel z nekim naključnim grafom, nato pa bo iterativno mutiral trenutni graf, tj. odstranil in dodal po eno povezavo tako, da bo graf še vedno kemijski
    ter izračunal vrednost $\sigma_t^{f(n)}$ za mutiran graf.
    Če mutirani graf izboljša vrednost $\sigma_t^{f(n)}$, ga agoritem sprejme kot trenutni graf.
    Po končnem številu iteracij bo algoritem vrnil stopenjsko zaporedje, kjer je bil dosežen maksimum, torej trenutnega grafa.

\end{enumerate}

\end{document}
